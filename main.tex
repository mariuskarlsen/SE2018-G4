\documentclass[11pt, twoside, openright]{report}
    % Alle pakkene kan leses om ved å google "CTAN PAKKENAVN".
    \usepackage[main=UKenglish, norsk, nynorsk]{babel} % Språkpakke
    \usepackage{microcode} % Muligens den beste pakken på markedet! Deaktiver den for å se hva som skjer :)
    \usepackage{ifthen, xkeyval, xparse, tikz, calc} % Noen nyttepakker som kreves av andre pakker.
    \usepackage{pdfpages, fancyhdr, caption, lipsum} % Importere pdf-er, bedre header og footer, caption-addon, lipsum generator(drafting).
    \usepackage[autostyle=false,norwegian=guillemets]{csquotes} % Setter riktige quotes.
    \usepackage{enumitem, multirow, multicol, longtable} % Liste, multirow for tabell, multicol for paragrafer, tabelpakke.
    \usepackage{booktabs, listings} % bedre kontrol på tabeller, gir mulighet for kodeeksempler.
    \usepackage{colortbl} % laster inn farger (Den laster også in xcolor- pakken automagisk)
    \usepackage[super]{nth} % Oh yes it can!
    \usepackage{makeidx} % For å generere en index

    % Drafting packages
    \usepackage[colorinlistoftodos,disable]{todonotes} % todo for å jobbe med
    \usepackage[nochapter, tocentry]{vhistory} % Versjonskontroll
    %\usepackage{layout} % Denne gir deg en egen side som forteller deg om nåværende marginer osv.
    % End drafting

% DOCUMENT SETTINGS
\addbibresource{reference.bib} % Denne forteller hvilke bibfil som skal benyttes (separer med komma).
% Disse gjør table of contents, table of figures og table of tables til multikolonne (ser bedre ut).
\renewcommand*{\multicolumntoc}{2}
\renewcommand*{\multicolumnlof}{2}
\renewcommand*{\multicolumnlot}{2}
\graphicspath{{assets/images}{assets/appendices}} % Setter paths til bilder for enkelhetens skyld når dere skal inkludere disse.
\newcounter{aCounterForSomething} % Dette er en variable (som kan kalles hva som helst) den vil telle opp hver gang den møtes i dokumentet.
\pagestyle{empty} % Vi renser header og footer
\pagenumbering{roman} % Denne setter sidetallvisning til romertall for frontmatter
\makeindex % Genererer index.

\begin{document}


% FRONTMATTER
    %------------------
    \pagestyle{empty}
    \pagenumbering{roman}
        %\input{sections/front/frontpage.tex}
    \cleardoublepage
    \setcounter{page}{3}

        %% !TEX root = ../../main.tex
\addcontentsline{toc}{chapter}{Abstract}
\begin{abstract}
	This is the abstract.
\end{abstract}

    \cleardoublepage
    \pagestyle{plain}

        %\addcontentsline{toc}{chapter}{Acknowledgements}
\section*{Acknowledgements}
    We would like to thank this and that.

        \cleardoublepage
        \setcounter{tocdepth}{1}
            %\begin{multicol}{2}
                \tableofcontents
                \listoffigures
                \listoftables
                \lstlistoflistings
            %\end{multicol}
        \cleardoublepage
        %\input{sections/front/authorship.tex}
    %------------------


% DRAFTINGMATTER (Bruk % forran for å deaktivere)
    %------------------

        %\layout %Printer layoutside for når dere skal justere margins osv (les om geometry pakken).
        \clearpage
        \begin{versionhistory}
    \footnotesize
  \vhEntry{Issue 01, Draft 01}{05.02.18}{OAD}{Boilerplate created}
\end{versionhistory} % Printer versjonshistorikk
        \listoftodos % Printer en oversikt over alle kommentarer som er skrevet med \todo[]{}

    %------------------ 

% MAINMATTER
    % --- mainmatter preamble ---
    \cleardoublepage
    \pagenumbering{arabic}
    \setcounter{page}{1}
    \pagestyle{fancy}

    %--- mainmatter content
        %\input{section/main/introduction.tex}
        %\input{section/main/literature.tex}
        %\input{section/main/prototype.tex}
        %\input{section/main/results.tex}
        %\input{section/main/reflection.tex}
        %\input{section/main/conclusion.tex}
    %------------------

% BACKMATTER
    %------------------

    \printbibliography
    \addcontentsline{toc}{chapter}{Bibliography}
    \cleardoublepage
    %\printindex % Denne printer index på ord.
    \addcontentsline{toc}{chapter}{Index}
    \cleardoublepage
    %\appendix
    %\input{sections/back/appendices.tex}
    %\cleardoublepage % HIOF Requirement
    \cleardoublepage           % HIOF Requirement

    %------------------

\end{document}