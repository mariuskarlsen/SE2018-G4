% !TEX root = ../../main.tex
%!TeX spellcheck = en-US
\subsection{Web application}
A web application, or shortened; \enquote{web app} is defined as \enquote{any application that uses a web browser as a client. The application can be as simple as a message board or a guest sign-in book on a website or as complex as a word processor or a spreadsheet} \cite{nations2016understanding}.
Since Internet has evolved from hosting simple {\small HTML} {\small(HyperText Markup Language)} pages with links of similar topics, to powerful tools such as spreadsheets, {\small EAS} {\small(Enterprise Application Software)} and games. It is important to distinguish between different technological aspects; Web apps can reside on any interconnected or local network, be it a intranet, {\small VPN} solution or the Internet \cite{joan2011difference}.
They are usually an integral part of a website, and requires a certain skillset to acheive; Websites can be produced as long as the developer has understanding of {\small HTML} and {\small CSS}. With web apps, this is not sufficient as it usually involves coding in a more difficult language: WebAssemby\footnote{auth0.com/blog/7-things-you-should-know-about-web-assembly}, Javascript\footnote{blog.thefirehoseproject.com/posts/why-is-javascript-so-hard-to-learn}, {\small PHP}\footnote{codingforums.com/php/234982-php-difficulty-compared-css.html} {\small (PHP: Hypertext Preprocessor)} or {\small ASP.NET} core\footnote{stackoverflow.com/questions/752035/prerequisite-knowledge-for-asp-net-mvc}, to name a few. It is also necessary to possess knowledge on server side scripting that connect and process data which should be displayed in the client software \cite{joan2011difference}.

\begin{figure}[htp]
    \centering
    \includegraphics[width=\textwidth]{placeholder}
    \caption{a model depicting a website and a webapp}
    \label{fig:literaturewebapp}
\end{figure}

Consequently, it is safe to assume that one, or a collective, of static webpages, with no further dependencies, is defined as a website. While a web application is considered running software, dependant on a network entity with webhosting\footnote{From wikipedia: a type of Internet hosting service that allows \\individuals and organizations to make their website accessible via the World Wide Web.} capabilities. Further research indicates that the concept web application has evolved into the more rigorous \enquote{web platform} dicipline. W3.org also reccommends the term web application where the purpose is to provide dynamic content to a user \cite{wwwc2016standards, wwwc2008webapplication}. Moreover, when introducing executable code in the environment which the web server processes, it is recommended to implement security measures to protect the client and server from malicous behaviour;

\subsubsection{Web application security}
Considering current trends and the increasing popularity of social interaction with web apps online, developers can find themselves cornered trying to provide reliable and fast web applications. It can be considered a sometimes futile effort from developers to secure their services while keeping up with this tremendous industry pace \cite{bhor2016analysisofweb}. \todo{More more more!}

\subsection{Gamification}
Different game techniques and mechanisms are frequently used to increase the learner's motivation and to enhance the learning process. This requires the development of an engaging environment equipped with necessary items to implement a competitive environment. In case of students, gamification is an effective approach widely used by schools to bring a positive change within students' behaviour and attitude and increase their overall learning process \cite{kiryakova2014gamification}. \par \vspace{1.5mm}
Gamification provides an incredible opportunity for teachers and learners to improve the current educational system. The phenomenon of gamification within education aligns and engaged the learning and entertainment together. Whereas the principles of gaming and entertainment are adapted by educational game designers to develop school curriculum and learning content \cite{kurshan2016gamification}. Therefore, gamification in education can be defined as, \enquote{the use of game mechanics and elements in educational environment} \cite[p.~2]{kiryakova2014gamification}.

\subsubsection{Gamification process}
The process of gamification within educational curriculum involves the incorporation of game-like-elements or game mechanics; such as self-elements and social-elements. The self-elements help a student to compete with oneself and realize self-achievement. For instance, points, levels, time restrictions, achievement badges, virtual goods, storyline, and aesthetics \cite{glover2013play, hanus2015assessing}. On the other hand, the social-elements are used to develop competition between students and their achievements badges/progress is made public. For example, leader-boards, virtual goods, interactive cooperation, and storyline  \cite{hanus2015assessing}. \par \vspace{1.5mm}
Both the types of game mechanics are used interchangeably to develop abilities and skills among students; such as 40\% of new skills and abilities can be transferred through game mechanics \cite{hsinyuanhuang2013gamification}. Besides, this integration of gamification and education is possible due to e-learning \cite{kiryakova2014gamification}. Additionally, the use of gamification in education has found to improve critical thinking, reasoning and problem-solving skills among students \cite{nemeth2015gamification}.

\subsubsection{Challenges}
Statistics shows that, 1/3 of high school students play video games for at least 3 hours per day {\small in some cases more than 3} \cite{kurshan2016gamification}. Therefore, there are several ways through, which schools teach or develop the concept of students with the incorporation of games; such as with the help of StarCraft, Pirates, Dungeons \& Dragons, and SimCity the concepts of artifact, Caribbean History, profitability, and engineering problems are transferred, respectively \cite{ericKlop2009gamification}. \par \vspace{1.5mm}
On the other hand, a number of challenges exists within the development of gaming based curriculum; especially, in terms of an enjoyable experience for the students. Moreover, the designed curriculum should develop cognitive learning; therefore, it is important that learning content should be interesting and motivating \cite{kurshan2016gamification}. Additionally, with the correct use of gamification techniques, teachers can convert non-intrinsically motivated activities into engaging activities \cite{nemeth2015gamification}.

\subsubsection{Educational use}
Allegedly, the use of gamification in educational system is a priority in today's changing systems; there is a lack of motivation, active participation, and engagement among students \cite{muntean2011raising}. Whereas teachers implement different techniques and approaches; such as use of information communication technology (ICT) to implement e-learning. With the use of e-learning techniques, teachers can track student's progress and software tools helps them to generate comprehensive reports \cite{deterding2012gamification, zichermann2011gamification, glover2013play}. \par \vspace{1.5mm}
Gamification in education, can evidently be implemented with the use of a strategical approach, seemingly important that a strategy should consists of the following main steps; determination of learners' characteristics, defining learning objectives, development of educational content and gamification activities, and addition of game mechanics/elements \cite{kiryakova2014gamification}. Limited research is however available on the practical use of gamification within the learning process for increased benefit \cite{kurshan2016gamification}. On the other hand, not all strategical approaches, techniques and theories of gamification are successful within the field of education. Thus, there is a need for more theories to develop multiple approaches to align gamification within education; such as the use of self-determination and peer-evaluation theory \cite{vanroy2017gamification}.

\subsubsection{Conclusion}
It is clear from the extensive research done in Gamification that the team will benefit from using some of these concepts. While some of them above fall outside the scope of this project, the continued use of badges and peer rating --- with the addition of scoreboards and statistics --- may greatly increase motivation to fullfill the course requirements.
